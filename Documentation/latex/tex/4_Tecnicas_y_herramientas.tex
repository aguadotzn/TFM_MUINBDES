\capitulo{4}{Técnicas y herramientas}

Esta parte de la memoria tiene como objetivo presentar las tecnologías y las herramientas de desarrollo que se han utilizado para llevar a cabo el proyecto. Se han estudiado diferentes alternativas de metodologías, herramientas, y se pretende aquí realizar un resumen de los aspectos más destacados, incluyendo comparativas entre las distintas opciones y, en caso de ser necesario, una pequeña justificación de las elecciones realizadas. 

\section{Metodologías}\label{metodologias}
En este apartado se describen las metodologías utilizadas para el desarrollo del proyecto.

\subsection{Estrategia de investigación}
En el presente proyecto se sigue una estrategia de investigación de análisis de datos cuantitativos. Y dentro de ésta se busca realizar una anális exploratorio y descriptivo que de pie a sacar una conclusiones de los datos escogidos como punto de partida

\begin{quote}
\textit{"La idea del análisis de datos es buscar patrones en los datos y sacar conclusiones. Existe una amplia gama de técnicas establecidas para analizar datos cuantitativos"} \cite{oates_researching_2006}. 
\end{quote}

Siguiendo esta estrategia se utilizan técnicas estadísticas exploratorias y descriptivas simples para encontrar patrones en los datos y comprobar si dichos patrones se encuentran realmente en los datos o si son sólo fruto del azar. Además, se utilizan tablas, mapas y gráficos para presentar los datos de una manera visual y sencilla. La recolección de datos se realiza mediante portales de datos abiertos vía descarga ordinaria. 

En cuanto a la realización del proyecto respecto de la parte técnica, se hace uso de \textit{Python} como lenguaje de programación de uso general y de \textit{R} como entorno y lenguaje de programación con enfoque estadístico. Además también he usado \textit{SQL} para realizar diversas pruebas. Para los gráficos me apoyo en diferentes herramientas \textit{open source} y par la visualización final en \textit{PowerBi}. También se realizará una pequeña web a modo de resumen para recopilar información relevante del proyecto. 

\subsection{Metodología de trabajo diaria: Scrum}\label{metodologias_scrum}
Scrum es un marco de trabajo de desarrollo de software que está dentro de las metodologías ágiles y es una de las más conocidas actualmente. 

Se trata de una herramienta muy útil en espacios donde los grupos de trabajo tienen dificultades para hacer las acciones u operaciones que les lleven a objetivos en común. Dicho de otro modo, \textit{Scrum} sirve para que equipos multidisciplinares trabajen en entornos complejos, donde los requisitos son muy cambiantes, y los resultados se tienen que obtener en un plazo corto de tiempo.  Se pueden observar las características principales en la figura \ref{ScrumSprint}.

\imagen{ScrumSprint}{\footnotesize{Resumen metodología \emph{Scrum}. Fuente: \url{manifesto.co.uk}}\label{ScrumSprint}}


En resumen, Scrum propone seguir un proceso de desarrollo iterativo e incremental a través de una serie de iteraciones denominadas \textit{sprints} y de revisiones. 
Hablando en términos personales, obtuve el \textit{título certificado de Scrum Master} el pasado mes de abril y es una metodología que aplico a diario durante el trabajo y que particularmente me apasiona. Si bien es cierto, siguiendo la teoría, que es necesario tener en cuenta que esta metodología fue pensada para trabajar en equipo por lo que en este trabajo se ha intentado emular de la mejor forma posible dadas las circunstancias: seguimiento de un \emph{board} (ver figura \ref{BoardExample1}), reuniones puntuales con los tutores, \emph{sprints} quincenales\ldots

\imagen{BoardExample1}{\footnotesize{\emph{Board} de tareas empleado durante el proyecto. Fuente:  Elaboración propia.}\label{BoardExample1}}

Se detallará más la metodología que se ha seguido en este proyecto en el \textbf{Apéndice A}.
\section{Herramientas}\label{herramientas}
En este apartado se describen las herramientas utilizadas para el desarrollo del sistema, concretamente en esta parte hago referencia a la parte software.

\subsection{Pycharm}\label{herramientas_pycharm}
\href{https://www.jetbrains.com/pycharm/}{Pycharm} es el editor es un IDE (Entorno de desarrollo integrado) desarrollado por la compañía Jetbrains, está basado en\textit{ IntelliJ IDEA}. Pycharm tiene cientos de funciones que lo puede ver como una herramienta muy pesada, pero que valen la pena ya que ayuda con el desarrollo del día a día. Es el entorno profesional por excelencia para trabajar con \textit{Python}. Es posible configurar también\textit{ jupyter notebooks} para trabajar de maner interna con este editor.

\begin{itemize}
	\item \subsubsection{Alternativas estudiadas}
	\begin{itemize}
		\item \href{www.sublimetext.com}{Sublime Text}: se pensó en esta alternativa debido a que se trata de un IDE bastante liviano en comparación con Pycharm. Finalmente se descartó por aspectos personales.
	\end{itemize}
\end{itemize}


\subsection{Jupyter Notebooks}\label{herramientas_jupyter}
\href{https://www.jetbrains.com/pycharm/}{Jupyter Notebook} es un entorno de trabajo interactivo que permite desarrollar código en \textit{Python} de manera dinámica, a la vez que integrar en un mismo documento tanto bloques de código como texto, gráficas o imágenes. Es un \textit{SaaS} utilizado ampliamente en análisis numérico, estadística y \textit{machine learning}, entre otros campos de la informática y las matemáticas. 

\subsection{Conda}\label{herramientas_conda}
\href{https://docs.continuum.io/anaconda/navigator/getting-started/ }{Conda} (Anaconda Navigator) es una interfaz gráfica de usuario para el gestor de paquetes y entornos de conda. A través del cuál se ha instalado \textit{ Jupyter Notebooks}.

\subsection{R studio}\label{herramientas_r_studio}
\href{https://www.rstudio.com}{R studio} es un entorno de desarrollo integrado (IDE) para el lenguaje de programación R, dedicado a la computación estadística y gráficos. Incluye una consola, editor de sintaxis que apoya la ejecución de código, así como herramientas para el trazado, la depuración y la gestión del espacio de trabajo. Aunque existen otros, sin duda es el IDE preferido para trabajar con R.

\subsection{Visual Studio code}\label{herramientas_atom}
\href{https://code.visualstudio.com}{Visual Studio code} es un editor de código fuente desarrollado por Microsoft para Windows, Linux y macOS. Incluye soporte para depuración, control de Git embebido, resaltado de sintaxis, finalización inteligente de código, fragmentos y refactorización de código. También es personalizable, para que los usuarios puedan cambiar el tema del editor, los atajos de teclado y las preferencias. Es gratuito y de código abierto, aunque la descarga oficial se realiza bajo licencia propia. Lo mejor de este editor es la gran cantidad de \textit{plugins} que puedes instalar facilitando programar en casi cualquier lenguaje.

\subsection{Firefox for developers}\label{herramientas_firefox}
Para el navegador se ha elegido Firefox en su versión para desarrolladores: \href{https://www.mozilla.org/es-ES/firefox/developer}{Firefox Developer Edition}.


\section{Herramientas de visualización de datos}\label{herramientasdataviz}
En este apartado se describen las herramientas utilizadas para el desarrollo la parte de visualización de datos.

\subsection{Power BI}
\href{https://powerbi.microsoft.com}{Power BI} es una solución de análisis empresarial que permite visualizar los datos y compartir información con toda la organización, o insertarla en su aplicación o sitio web. En este caso se ha utlizado a través de una máquina virtual windows proporcionada por la universidad.

\begin{itemize}
	\item \subsubsection{Alternativas estudiadas}
	\begin{itemize}
		\item \href{https://www.qlik.com/us}{Qlik}, fue visto y utilizado durante el máster y es sin duda una gran alternativa aunque me encuentro más cómodo trabajando con PowerBI.
		\item \href{https://www.tableau.com}{Tableau}, esta fue nuestra  segunda opción. Tableu es una herramienta de visualización muy potente que permite realizar visualizaciones dinámica muy completas. Finalmente fue descartado por problemas de hardware. 
	\end{itemize}
\end{itemize}

\subsection{CartoDB}
\href{https://carto.com}{CartoDB} es una de las empresas líderes en cartografía mundiales. Es un servicio de pago aunque actualmente se ha accedido a él para este trabajo mediante una licencia universitaria que posee github en la que se tiene acceso a diversas herramientas entre las que forma parte \textbf{CartoDB}. Ha sido utilizado para la distribución geográfica de las estaciones.

\begin{itemize}
		\item \href{https://www.mapbox.com/}{Mapbox}  es una herramienta que permite geolocalizar posiciones entre otras opciones. Es de pago aunque tiene una licencia de uso libre restringida por número de peticiones al servidor. También se ha usado para la visualización final pero se han encontrado contratiempos durante el desarrollo.
\end{itemize}

\subsection{Vega}
\href{http://vega.github.io/}{Vega} es una herramienta totalmente \textit{open source} que permite, a través de la ingesta de diferentes formatos, crear gráficos interactivos. 

\begin{itemize}
		\item \href{http://vega.github.io/}{Voyager2} es una herramienta interna de vega que, dado un dataset, permite realizar un análisis exploratorio de las variablas que lo forman a través de diversos gráficos.
\end{itemize}

\subsection{DataWrapper}
\href{https://www.datawrapper.de}{DataWrapper} es una herramienta que dado un dataset (generalmente ya preprocesado ya que no se trata de una herramiente de limpieza sino que se enfocan más en la parte de visualizado) permite generar gráficos, mapas o tablas con el objetivo principal de enriquecer contenido en historias escritas. Principalmente está pensado para el acompañamiento de parte gráfica a noticias en diferentes medios de comunicación.


\section{Tecnologías}\label{tecnologias}
 En este apartado se describen las tecnologías utilizadas, concretamente lenguajes de programación, para el desarrollo del proyecto. 
 
 \subsection{Python}\label{tecnologias_python}
 
 \href{python.org}{Python} es un lenguaje de programación interpretado cuya filosofía hace hincapié en una sintaxis que favorezca un código legible. Los programas escritos en\textit{Python} no necesitan compilarse de antemano para poder ejecutarse, por lo que es fácil probar pequeños fragmentos de código y hacer que el código sea más fácil de mover entre las plataformas.
 
 
Este lenguaje tiene licencia de software libre y dispone de una gran comunidad de usuarios que se encarga de enriquecerlo mediante la creación de nuevas librerías, funciones, etc. 

Python es muy popular debido a su sencilla integración con otras plataformas y a que dispone de gran cantidad de librerías. Algunas librerías usadas en este proyecto han sido \href{https://ggplot2.tidyverse.org/}{ggplot}, \href{https://numpy.org/}{numpy}, \href{https://numpy.org/}{urllib}, \href{https://pandas.pydata.org/}{pandas} o \href{https://bokeh.org/}{bokeh} entre otras.

  \subsection{R}\label{tecnologias_R}
  \href{r-project.org}{R} es un entorno y un lenguaje de programación enfocado en el análisis estadístico de los más utilizados en el campo de la minería de datos que pueden aplicarse a gran variedad de disciplinas. De nuevo es software libre y es uno de los lenguajes de programación más utilizados en investigación científica.
  
  Algunas librerías usadas en este proyecto han sido \href{https://ggplot2.tidyverse.org/}{ggplot}, \href{https://numpy.org/}{numpy}, \href{https://numpy.org/}{urllib}, \href{https://pandas.pydata.org/}{pandas} o \href{https://bokeh.org/}{bokeh} entre otras.
  
   \hl{Librerias R}

 \subsection{Tinybird}
\href{https://tinybird.co//}{Tinybird.co} es un \textbf{aplicación en beta} que gracias a un cúmulo de circustancias hemos tenido la suerte de probar en este proyecto. Se trata de una aplicación que utiliza lenguaje \textit{SQL} para realizar consultas sobre datasets. Es decir, dado un \textit{dataset} se crean diversos \textit{pipelines}, esa es la denominación que le dan, con los que podemos analizar, transformar y limpiar los datos de acuerdo a nuestros objetivos. Parece una especie de \textit{notebook} de jupyter aunque no lo es dado que no posee tal cantidad de opciones. 

Una vez tenemos los datos preprocesados podemos establecer un \textit{endpoint} a través del cuál es posible acceder de manera externa mediante \textit{API REST}. Después es muy fácil consumirlo desde, por ejemplo, un apicación web llamando a ese endpoint y acediendo a los datos ya limpios y listos para visualizarlos. 

\section{Documentación}\label{docs}
En este apartado se describen algunas de las herramientas utilizadas para la parte de la documentación del proyecto.

 \subsection{La\TeX}\label{docs_latex}
  La\TeX es un sistema de composición de textos, orientado a la creación de documentos escritos que presenten una alta calidad tipográfica. Por sus características y posibilidades, es usado de forma especialmente intensa en la generación de artículos y libros científicos. 
  
   \begin{itemize}
  	\item \subsubsection{Detalles}
  	\begin{itemize}
  		\item \href{https://www.texstudio.org}{\textit{TeXstudio}}: como editor de escritorio de La\TeX se ha utilizado \textit{TeXstudio}, una recomendación personal de Carlos, tutor académico, y que sin lugar a dudas a ha supuesto un gran salto de calidad a la hora de realizar el trabajo en cuanto a la parte de la memoria.
  		\item \href{http://www.tablesgenerator.com/latex_tables}{\textit{Tables\_generator}}: una de las cosas más engorrosas a la hora de usar La\TeX\  son la tablas. Por esa razón se ha empleado \textit{tables\_generator} para realizar esta parte de una manera más cómoda. 
  	\end{itemize}
  \end{itemize}
  
 \subsection{Zotero}\label{docs_zotero}
 \href{https://www.zotero.org/}{Zotero} es un gestor de referencias bibliográficas, libre, abierto y gratuito desarrollado por el \textit{Center for History and New Media} de la Universidad George Mason que funciona también como servicio. El funcionamiento de zotero es sencillo y está basado en los principios: recopilar, organizar, citar, sincronizar y colaborar. En la figura \ref{zotero} podemos ver la aplicación de escritorio para Mac.
  
  \imagen{zotero}{\footnotesize{Pantalla principal Zotero. Fuente: elaboración propia}\label{zotero}}
  
   \begin{itemize}
   	\item \subsubsection{Alternativas estudiadas}
   		\begin{itemize}
	 	\item \href{https://www.mendeley.com/}{Mendeley}  es una aplicación web y de escritorio, propietaria y gratuita. Permite gestionar y compartir referencias bibliográficas y documentos de investigación, encontrar nuevas referencias y documentos y colaborar en línea. Fue descartada debido a problemas en otros trabajos anteriores con la sincronización entre dispositivos.
	   	\item \href{https://www.zenhub.com/}{Zenhub} es otro gestor de referencias bibliográficas que descarté por experimentar problemas con la extensión del navegador.
		\end{itemize}
	\end{itemize}
  
  \section{Otras herramientas}\label{otrasherramientas}
En este apartado se describen otras herramientas que se han empleado también durante el proyecto.

 \subsection{Git}
Git \cite{git_2019} es un sistema de control de versiones. Este tipo de sistemas registran los cambios realizados sobre un archivo o conjunto de archivos a lo largo del tiempo, de modo que es posible recuperar versiones específicas del mismo archivo más adelante.

 \subsection{Github}
\href{https://github.com/}{Github} es una plataforma cloud de desarrollo colaborativo de software para alojar proyectos utilizando el sistema de control de versiones \textit{git} \cite{git_2019}. El código se almacena de forma pública, aunque también se puede hacer de forma privada, además desde principios de este mismo año las opciones de repositorios privados se permiten de manera gratuita.  GitHub aloja tu repositorio de código y te brinda herramientas muy útiles para el trabajo en equipo. 

Se ha utilizado esta herramienta a modo de control de versiones y también se ha usado el \textit{project board} interno como gestor de tareas y de \textit{sprints}. Los \textit{project boards} funcionan, generalmente por repositorio y se pueden utilizar para crear flujos de trabajo personalizados que se adapten a las necesidades de cada proyecto.

\begin{itemize}
	\item \subsubsection{Alternativas estudiadas}
	\begin{itemize}
		\item \href{https://www.zenhub.io/}{Zenhub} es una extensión de Chrome para github. Se utiliza para gestionar proyectos y funciona de manera nativa en la interfaz. Se basa en la metodología ágil y resulta verdaderamente útil a la hora de realizar y gestionar un proyecto.
		\item \href{https://www.trello.com/}{Trello} es un gestor de tareas que permite el trabajo de forma colaborativa mediante tableros compuestos de columnas  que representan distintos estados. Se basa en el método \textit{Kanban} para gestión de proyectos, con tarjetas que viajan por diferentes listas en función de su estado. 
	\end{itemize}
\end{itemize}

\subsection{Github pages}
Dentro de esas herramientas que se mencionaban al describir Github en el apartado anterior se encuentran \textbf{github pages}. Esta herramienta permite alojar sitios web estáticos gratuitamente dentro un repositorio de código. GitHub pages permite dos modalidades de publicación:
\begin{enumerate}
	\item La primera es “User site” (solo se podrá tener un sitio de este tipo por cuenta); en este caso el sitio web será publicado en username.github.io (siendo username el nombre de usuario de la cuenta).
	\item La segunda opción es “Project site” (proyectos ilimitados) el cual será publicado en \textit{username.github.io/repository} (siendo repository el nombre del repositorio).
\end{enumerate}

Es una manera sencilla, rápida, cómoda de realizar pruebas para una web. Para proyectos de carácter académico en la que mostrar unos resultados en una opción bastante recomendada.

\begin{itemize}
	\item \subsubsection{Alternativas estudiadas}
	\begin{itemize}
		\item Plataformas en la nube:  como por ejemplo como Amazon AWS u OpenShift son muy populares hoy en día y permiten prácticamente lo mismo que github pages.
		\item \href{https://www.heroku.com/}{Heroku}  es un servicio de almacenamiento en la nube que además tiene mecanismos y herramientas para que la puesta en producción de las aplicaciones web sea prácticamente automática.
	\end{itemize}
\end{itemize}
    
 \subsection{StackOverFlow}

\href{https://stackoverflow.com/}{Stack Over Flow} es una de las comunidades de desarrolladores más importantes del mundo en la que se responden cuestiones de diferentes lenguajes. Es una herramienta fundamental para programadores. Actualmente Python es el lenguajde programación con más consultas en la plataforma.

 \subsection{OpenRefine}

\href{https://github.com/OpenRefine/OpenRefine}{OpenRefine} es una potente herramienta \textit{open source} para trabajar con datos: limpiarlos, transformarlos de un formato a otro y ampliarlos con servicios web y/o datos externos. Se utilizó para categorizar y ver los datos en la parte de investigación. En la figura \ref{openrefine} una imagen de la herramienta.

  \imagen{openrefine}{\footnotesize{Detalle vista principal. Fuente: elaboración propia}\label{openrefine}}

