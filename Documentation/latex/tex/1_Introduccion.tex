\capitulo{1}{Introducción}

La contaminación ambiental se ha incrementado, y sigue haciéndolo a un ritmo vertiginoso, de manera preocupante en los últimos años. Constituye uno de los problemas más serios a los que se enfrenta el ser humano. Hoy día ya no es una cuestión localizada en algunos lugares sino que el viento se ha encargado de convertirlo en un problema global. Los gases provenientes de automóviles, camiones, procesos industriales, sistemas de calefacción e incluso hasta el humo de los cigarrillos de miles de fumadores se juntan para contaminar el aire que consumimos a diario. 

Respirar aire limpio y sin riesgos para la salud debería ser un derecho de toda persona. Está demostrado que la contaminación atmosférica causa graves daños a la salud y al medio ambiente. Los niveles actuales de contaminación atmosférica provocan la muerte de entre cuatro y cinco millones de personas por exposición directa al aire contaminado en todo el mundo \cite{sharma_anirudh_anirudh_2018}. Si hablamos de España alrededor de 16.000 muertes prematuras en España \cite{informe_EeA_2019}.

Pero la contaminación es un tema realmente amplio, dado que la mayor parte de contaminantes son expulsados por procesos industriales y automóviles; y además éstos se concentran principalmente en grandes urbes, es precisamente ahí donde debemos realizar nuestros mayores esfuerzos. Casi la mitad de la población mundial vive actualmente en ciudades, y para el año 2050 se prevé que aumente a un 75\% \cite{bbc_ciudades}. Nuestro reto ahora es el intentar cambiar la manera en la que vivimos en aquellos puntos de mayor concentración de contaminación para evitar seguir contaminando de la manera que lo hacemos. Globalmente ya se están tomando medidas, diversos países proponen medidas para tratar de frenar o reducir la contaminación. 

En el caso concreto de la ciudad de Madrid, se trata de una de las dos ciudades españolas, junto con Barcelona, que está obligada a cumplir los niveles máximos de \ce{NO2} impuestos como medidas para reducir la contaminación dictados por la unión europea. 

En las páginas siguientes vamos a realizar un estudio mediante el cuál analizaremos la calidad de aire de la ciudad de Madrid gracias al portal de datos abiertos y basándonos en los niveles de calidad del aire que recomienda la OMS como normales.