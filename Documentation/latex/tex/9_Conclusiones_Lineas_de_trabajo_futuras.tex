\capitulo{9}{Conclusiones y Líneas de trabajo futuras}

En está última sección se exponen las conclusiones finales derivadas del trabajo. De la misma manera abordaremos las línea de trabajo futuro.


\section{Conclusiones proyecto}\label{conclusiones_proyecto}
Se han cumplido los objetivos marcados al inicio del proyecto. Se han explorado los datos del ayuntamiento de Madrid de un punto de vista analítico y mediante diversas técnicas con objetivos comunes. Además se ha realizado un \textit{dashboard} de apoyo a la investigación así como una pequeña web con las partes que nos han parecido más interesantes.

Como parte crítica al proyecto creo que hubiera sido interesante cruzar otros tipos de datos. Como se ha nombrado  anteriormente la contaminación del aire se basa en unas mediciones únicas sino que éstas se ven afectadas por diversos factores como por ejemplo la velocidad del viento, las precipitaciones o las temperaturas. También hubiera sido interesante cruzar los datos del tráfico o los precios de los combustibles para ver de qué manera afectan éstos.


\section{Conclusiones derivadas del estudio}\label{conclusiones_proyecto}
La contaminación en Madrid es un problema grave, así lo reflejan los datos y así nos lo recuerdan desde Europa. Se deben tomar medidas urgentes para intentar rebajar los niveles, sobre todo de los gases más nocivos para los seres humanos, antes de encontrarnos ante un problema irreversible.

La comunidad madrileña es la región del mundo, solo por detrás de Singapur, que tiene más metros de autovía por ciudadano. A pesar de ser una ciudad diseñada para el vehículo privado y su fomento, cada vez más los ciudadanos están concienciados ante un problema que acarrea problemas importantes. Medidas desde el ayuntamiento para restringir el tráfico a los puntos más contaminados pueden y deben ser una solución ante los altos índices de gases tóxicos. Sin embargo, los mayores cambios para que realmente se produzcan alteraciones bruscas en éstos niveles de contaminación se consiguen alterando los hábitos de la gente para conseguir una ciudad más limpia, segura y ecológica.

El análisis y comprensión de toda esta cantidad ingente de datos puede ofrecernos una mejor perspectiva de por qué se produce esta contaminación del aire y cuáles pueden ser las medidas más eficaces y efectivas para luchar contra este tipo de problemas. Pero la responsabilidad seguirá siendo nuestra, tanto a nivel personal como colectivo.

Aunque el Big Data nunca reemplazará la responsabilidad ambiental, sí que puede proporcionar las herramientas y la visión que necesitamos para reducir la contaminación y mejorar la calidad del aire.


\section{Conclusiones personales}\label{conclusiones_personales}
Desde un punto de vista personal me encuentro satisfecho con el trabajo realizado. Después de realizar este tipo de trabajos siempre nos queda la duda de por qué si conocemos y tenemos los datos no hacemos más para intentar poner fin a este tipo de problemas. Creo que los seremos humanos solo vemos los problemas cuando los tenemos encima. No somos capaces, o nos cuesta mucho,  pensar medidas a largo plazo.

\section{Líneas de trabajo futuras}\label{lineas_futuras}

Algunos itinearios a seguir dentro de investigaciones en la misma línea serían los siguientes. 

\begin{itemize}
	\item Aplicación de Inteligencia Artificial, modelos de \textit{machine learning}, para predecir los valores de contaminación futuros y anticiparse a los escenarios. Dado que el máster tiene una carga importante de este tipo de técnicas esta era una de las opciones  a incluir dentro de este proyecto Debido a la falta de tiempo no se ha podido realizar.
	\item Integración en el estudio de otro tipo de datos, tráfico, clima o precios del gasóleo son algunos ejemplos, con los datos de contaminación para enriquecer los resultados. La precisión del estudio aumentaría ya que son variables que afectan directamente a los niveles. 
	\item Otra línea de trabajo puede ser el estudio de concienciaciación de la población ante este tipo de problemas y el cómo hacerles llegar esta información a través de la tecnología. ¿Qué medios técnicos son los mejores y por qué?, ¿a qué se debe que no funcionen las medidas de concienciación existentes?, ¿cómo podemos ayudar a las instituciones para ser más efectivos cuando se dan episodios de contaminación?. Sin duda es una parte compleja pero necesaria para que la población conozca el problema. Aplicaciones para trasladar este tipo de datos a la ciudadanía resultan muy útiles.
\end{itemize}



