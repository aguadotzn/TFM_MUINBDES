\capitulo{6}{Trabajos relacionados}

En esta sección se han recopilado algunos recursos  útiles que tratan sobre el mismo tema o temáticas similares. Realmente hay muchos estudios sobre la calidad del aire. Quizás el hecho de tener los datos en modo \textit{open data} da más libertad para usar estos datos y realizar proyectos con ellos.

\section{Trabajos útiles}

\begin{itemize}
	\item \href{https://github.com/UlisesGascon/Charla-AireMadrid-los-datos-abiertos}{\textit{AireMadrid} 'La realidad de los datos abiertos'} (Ulises Gascón, 2017): Ulises nos habla de las dificultades que supone crear un proyecto \textit{Open Source} que utiliza los datos abiertos de calidad del aire que publica el Ayuntamiento de Madrid para informar y permitir la reutilización efectiva de esa información por parte de los usuarios y otros desarrolladores.
	\item \href{https://github.com/Toolson12/poisonMAD}{\textit{Respira Madrid}} (José M. Martínez, 2019): Implementación en Java para el tratamiento e interpretación de datos sobre la calidad del aire en Madrid.
	\item \href{https://github.com/chucheria/CalidadAire}{\textit{Estudio calidad aire Madrid}} (Beatriz@Chucheria, 2016): Análisis sobre la calidad del aire y predicción sobre los niveles de contaminación de la ciudad de Madrid.
	\item \href{https://www.rodrigodemiguel.es/AnalisisAireMadrid/}{\textit{Análisis de la contaminación y la calidad del aire en Madrid}} (Rodrigo de Miguel González, 2016): análisis detallado del \ce{NO2} del aire de la ciudad de Madrid, compuesto que utiliza el Ayuntamiento para poner en marcha los periodos de restricciones por alta contaminación, comprobando el resultado de dichos periodos a lo largo del año.
	\item \href{https://idus.us.es/xmlui/handle/11441/79499}{\textit{Big data aplicado al transporte y en las ciudades: adaptación a la ciudad de Sevilla}} (Alejandro Casado Reinaldos, Trabajo Fin de Grado, Universidad de Sevilla, 2018): estudio analítico de datos de la ciudad de Sevilla. 
	\item \href{http://oa.upm.es/47313/}{\textit{Análisis temporal multivariante de la contaminación atmosférica dentro del distrito de metropolitano de Quito}} (Juan Luis Manosalvas Paredes, Trabajo Fin de Máster, Universidad Politécnica de madrid, 2017): se trata de un estudio que tiene  similitudes en cuanto a que se realiza un estudio de la contaminación, se presentan las misma partículas y contaminantes, pero resulta curioso ver como varían las legislaciones al tratarse de un país de América Latina.
	\item \href{https://github.com/Fictizia/aireMAD}{\textit{AireMAD}} (Fictizia escuela, 2017): aplicación libre desarrollada por Fictizia (escuela de programación) para poder ver los datos de la calidad del Aire de Madrid en tiempo real. Actualmente no se encuentra en desarrollo pero tiene recursos interesante y una robusta documentación disponible  \href{https://github.com/Fictizia/aireMAD/wiki}{aquí}.
\end{itemize}


 
