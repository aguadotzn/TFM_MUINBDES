\capitulo{2}{Objetivos del proyecto}

En este apartado se van a detallar los diferentes objetivos que se buscaban con la realización de este trabajo fin de máster.

Se parte de la hipótesis de que es posible obtener datos abiertos que permitan caracterizar  los niveles de contaminación de la ciudad de Madrid y que a partir de estos datos se puede crear un modelo para obtener, analizar y visualizar los datos 
de la calidad del aire.

\section{Objetivos}\label{objetivos-generales}

\begin{itemize}
\tightlist
\item Analizar, usando los datos abiertos que proporciona el Ayuntamiento de la ciudad de Madrid, la calidad del aire de la ciudad, poniéndola en relación con los objetivos planteados por la OMS.
\item Preparación de los datos (Selección, limpieza, transformación, preprocesado)
\item Visualización de datos que ayude a la respuesta de diversas preguntas analíticas 
\item Respuesta de diversas preguntas analíticas
\item Elección de unos contaminantes para analizar en detalle
\end{itemize}

\section{Objetivos personales}\label{objetivos-personales}
\begin{itemize}
\tightlist
\item
 Realizar un trabajo fin de máster más enfocado a la investigación. Durante trabajos académicos anteriores siempre había partido de una base mas técnica con unos objetivos finales puramente técnicos. En este caso la combinación de una parte técnica con una parte más de exploración o investigación hizo que el alumno se decantara por esta opción. La base del máster proporcionaba casi todo lo necesario para aplicar las técnicas aprendidas con un bonito reto a explorar. Además no descarto el realizar un doctorado en un futuro y creo que puede ser un acierto  comenzar a tener esta visión más de investigación.
 \item
 Realización de un proyecto que involucre datos reales
\item
  Mejorar conocimientos de Python, R, visualización de datos y estadística. 
  \item Profundizar en el campo del \textit{data science}.
\end{itemize}
