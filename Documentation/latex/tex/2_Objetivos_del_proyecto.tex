\capitulo{2}{Objetivos del proyecto}

En este apartado se van a detallar los diferentes objetivos que se buscaban con la realización de este trabajo fin de máster.

Se parte de la hipótesis de que es posible obtener datos abiertos que permitan caracterizar  los niveles de contaminación de la ciudad de Madrid y que a partir de estos datos se puede crear un modelo para obtener, analizar y visualizar los datos 
de la calidad del aire.

\section{Objetivos}\label{objetivos-generales}
Comenzaremos con los objetivos generales del proyecto. 
\begin{itemize}
\tightlist
\item Analizar, usando los datos abiertos que proporciona el Ayuntamiento de Madrid, la calidad del aire de la ciudad, poniéndola en relación con los objetivos planteados por la Unión Europea
\item Recopilación, descripción, exploración, visualización
\item Preparación de los datos (Selección, limpieza, transformación, preprocesado)
\item Visualización de datos que ayude a la respuesta de diversas preguntas analíticas 
\end{itemize}

\section{Objetivos personales}\label{objetivos-personales}
Los objetivos personales que he perseguido durante todo el desarrollo han sido los siguientes:
\begin{itemize}
\tightlist
\item
 Realizar un trabajo fin de máster más enfocado a la investigación. Durante mi trabajo fin de grado realicé una aplicación web y móvil. Podemos decir que aquel trabajo era algo más puramente técnico por lo que cuando vi la posibilidad de algo que combinara la parte de exploración, al final ha sido necesario involucrarse e indagar en la parte científica del trabajo, con la parte de tecnología, todo lo aprendido en el máster, lo escogí sin dudar. Además no descarto el realizar un doctorado en un futuro y creo que me va venir bien comenzar a tener esta visión más investigadora.
 \item
 Realización de un proyecto que involucre datos reales
\item
  Mejorar mis conocimientos de Python, R, visualización de datos y estadística. Dado que mi trabajo actual engloba el desarrollo web, he escogido esos lenguajes y esta particularidades por que deseo enfocar mi futuro profesional hacia este campo.
   \item Profundizar en ciencia de datos
  \item Profundizar en visualización de datos
\end{itemize}
