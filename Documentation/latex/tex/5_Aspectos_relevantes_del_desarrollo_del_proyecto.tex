\capitulo{5}{Aspectos relevantes del desarrollo del proyecto}

Este apartado pretende recoger de manera teórica y breve los aspectos más interesantes del desarrollo del proyecto así como su justificación: decisiones que se han tomado, avances evolutivos por tipo y problemas que surgieron durante toda la realización del proyecto.


\section{Elección del proyecto}\label{inicio-proyecto}
Cuando llegó la hora de buscar proyecto, entre las diferentes opciones existían proyectos de carácter más técnico, también orientados a temas mas puramente teóricos o de carácter más de investigación. Sin duda, ésta última fue la opción escogida por el reto que esto supone. No existe por lo tanto un ruta exacta a seguir en este tipo de trabajos, sino que se trata más de tener un objetivo concreto pero a sabiendas de que el camino nos va a conducir a sacar unos patrones de los datos que pueden ser sorprendente o incluso inesperados. 

\section{Formación}\label{formacion}

Toda la formación con la que se ha contado para realizar este trabajo se ha adquirido gracias al máster cuyo proyecto final representa este trabajo. En éste se han aprendido las bases suficientes como para realizar un trabajo de éstas características. Si bien es cierto que la motivación principal era conducir la carrera del alumno hacia el ámbito del \textit{data science} y no ha habido demasiada formación de \textit{R} durante el máster. Es por ello que para esta funcionalidad concreta se han realizado un par de cursos extra. Para concretar algo más los cursos que se han tomado han sido de la plataforma \href{https://openwebinars.net/}{Open Webinars}.

\section{Metodologías/estrategias aplicadas}\label{metodologia}

Ya se han nombrado las metodologías más usadas durante el proyecto: por un lado \textit{Scrum} como metodología de desarrollo iterativa y por otro la estrategia de investigación ha sido la de análisis de datos cuantitativos. 

La estrategia de investigación a permitido aliarse perfectamente a la metodología escogida, al desarrollo del código fuente y a la exploración de la parte científica y legislativa del proyecto.

En resumen, ambas decisiones han fomentado de manera positiva tanto la motivación como el progreso del proyecto.


\section{Desarrollo análisis datos}
El modelo empleado para el desarrollo del proyecto comprende la exploración, transformación y posterior visualización de datos, se detalla de manera precisa en el apartado 7. 

Hay que destacar aquí que cuando hablamos de Big Data estamos hablando de cantidades realmente enormes pero en este caso, aún con todo los datos descargados de todos los años, no supone una cantidad excesivamente grande de datos. Es por eso que se han elegido este tipo de lenguajes de programación, quizás más enfocados también a otras áreas pero perfectamente válidos para un estudio de este tipo.

 \hl{SEGUIR AQUI}
 
\section{Desarrollo visualización}\label{grafico}
La parte de visualización comprende varias partes:

\begin{itemize}
	\item La parte final del \textit{dashboard}, realizada con \textit{PowerBI}.
	\item Todas las gráficas internas que se han realizado en los diferentes lenguajes de programación, con ayuda de diferentes bibliotecas, y con la finalidad última de explorar los datos o ayudarse a las explicaciones.
	\item El uso de otras herramientas, ya nombradas, mediante las cuales también se han realizado gráficas. Ya sea de apoyo a la memoria, como explicación añadida web o para demostrar ciertas afirmaciones.
\end{itemize}

En general ha sido una parte costosa en términos de tiempo. Además se han intentado seguir guías de estilo \cite{visualisation_2016} adecuadas o tener en cuenta gamas de colores aptas realizando la visualización de la mejor manera posible. 

\section{Desarrollo web}\label{dweb}

Como complemento a la visualización se ha realizado también una pequeña web, alojada en \textit{github pages}, que pretende ser un pequeño resumen vertical de todo lo que se ha estudiado en el proyecto.

\section{Documentación}\label{docs}

La documentación se ha realizado de manera progresiva siempre que se podía se añadían partes de la misma y siempre preguntando sobre las cosa que no estaban claras a los tutores, que han ayudado convenientemente a la realización de la misma. Se han realizado dos envíos a los tutores antes del envío final para su corrección .

\section{Dificultades encontradas}\label{inicio-proyecto}

Durante el desarrollo 

Una de las partes más importantes del proyecto, en cuestión de tiempo, ha sido la de familiarizarse con los conceptos científicos que requería el mismo. 

 \hl{SEGUIR AQUI}


\section{Agradecimientos}\label{reconocimientos}
Me gustaría añadir este pequeño apartado ya que afortunadamente han sido muchas las personas que se han visto involucradas en el proyecto de manera indirecta. 
\begin{itemize}
	\item \textit{Jorge Gomez}, \textit{Javi Santana} (fundadores de \textbf{tinybird}): además del acceso a la \textit{beta} de la aplicación se han prestado a responder de manera altruista todas las preguntas que he realizado sobre la misma.
	\item \textit{José Amador Viejo} (Director general de control ambiental, \textbf{Ayto de Madrid}): ha respondido sin ningún problema a las dudas sobre legislación que se han realizado.
	\item \textit{Juan Barcena del Rieg} (Portavoz de \textbf{Ecologistas en acción}): se han intercambiado varios correos, sobre todo para la parte legislativa y también de la parte ambiental.
	\item \textit{Javier Di Deco Sampedro} (Data Scientist en \textbf{Piperlab}): existe un \href{https://twitter.com/datoxnitro_bot}{bot de twitter} realizado por la mencionada empresa que publica en tiempo real los niveles de contaminación por dióxido de nitrógeno de la ciudad de Madrid. Se envió un mensaje a la empresa para aprender las metodologías empleadas en el desarrollo del \textit{bot} y Javier me atendió de manera amable y eficaz.
	\item \textit{Brenda Valverde}, actual periodista de \textbf{Newtral} mantuvimos una interesante conversación de carácter periodístico sobre contaminación ambiental en general y Madrid central en particular de la que se obtuvieron diversas fuentes para recopilar información. Es posible seguir su cuenta de twitter \href{https://twitter.com/brbrebren}{aquí}.
	\item \textit{Carlos E.Vivaracho Pascual} (tutor académico, \textbf{Universidad de Valladolid}): siempre atento durante el desarrollo del proyecto y poniendo las cosas fáciles cuando había incertidumbre.
    \item \textit{Fernando Cuenca Cabezas} (tutor de empresa, \textbf{Minsait}): cómplice de algunas de las decisiones del proyecto siempre se ha prestado a la colaboración durante todo del desarrollo. Además ha resultado fundamental para abordar diversas cuestiones técnicas.
\end{itemize}

