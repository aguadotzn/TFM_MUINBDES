\capitulo{8}{Resultados}

En está penúltima sección se exponen las conclusiones derivadas del trabajo. 



\section{Preámbulo}
Los resultados son el remate final a este estudio, sin embargo, quiero hacer aquí una pequeña aclaración. Los efectos de este estudio realizado por otro conjunto de personas y partiendo de los mismos datos pueden ser diferentes dado que el enfoque que se le da es distinto. Por ejemplo yo me he enfocado más en un contaminante, \ce{NO2}, y en éste comparado a los niveles de la OMS. Si bien es cierto que se ha encontrado patrones similares con otros trabajos (ver más en la sección 6) no era este el objetivo final. 

En la figura \ref{diagramtech} se puede ver el conjunto de desarrollo completo que se ha seguido para llegar a las visualizaciones.

  \imagen{diagramtech}{\footnotesize{Diagrama desarrollo. Fuente: Elaboración propia.}\label{diagramtech}}
  
  Los resultados que se exponen en el siguiente apartado son los que se han obtenido de observar, analizar y estudiar esas visualizaciones para dar respuesta a las preguntas analíticas. 
  

\section{Resultados generales obtenidos}
Los resultados a los que hemos llegados son los siguientes:

\begin{itemize}
	\item La contaminación
\end{itemize}

\section{Preguntas analíticas}

Algunas de las preguntas analíticas, marcadas al inicio del proyecto, han sido las siguientes:

\begin{itemize}
	\item La contaminación
	\item Los protocolos de anticontaminación se basan en los niveles de los óxidos de nitrógeno. Estos niveles suben por la noche por lo que siempre se espera a última hora para activar dichos protocolos. Es una afirmación correcta ¿
	
	\item Existen contaminantes que puedan estar asociados a temporalidad ?
	
	\item Consideras que las últimas medidas aplicadas por el Ayuntamiento de Madrid durante los últimos años han mejorado la calidad del aire? Que contaminantes? Alguna zona/estación en concreto ? Puedes confirmar esta información contrastando la evolución temporal del número de alertas, avisos y preavisos en las diferentes zonas ?
	
	\item Qué área es la más contaminada ?
	
\end{itemize}



 Acuérdate de por que hemos empezado primero reuniendo los datos y analizándolos: para buscar conocimientos útiles o valiosos dentro de todos estos conjuntos de datos, para responder a preguntas o para mejorar los procesos de negocio. Por ejemplo, ¿Debemos cambiar algo en nuestro proceso para eliminar cuellos de botella?, ¿Deberíamos añadir datos a la aplicación para que sea más precisa?, ¿Debemos segmentar nuestra población en grupos mejor definidos para tener un marketing dirigido más eficaz? Este es el primer paso para convertir el conocimiento en acción.

Una vez que hayamos decidido como actuar, el siguiente paso es averiguar como implementar la acción. ¿Que es necesario para añadir esta acción a nuestro proceso o aplicación? ¿Como vamos a automatizarla? Debemos identificar a los grupos de interés y hacer que se involucren en este cambio.

Al igual que sucede con cualquier optimización de procesos, tenemos que monitorizar y medir el impacto de la acción en el proceso o aplicación. Evaluar el impacto conlleva una evaluación de resultados.

La evaluación de resultados de la acción aplicada determinará los pasos a seguir: ¿Necesitamos llevar a cabo un análisis adicional con el fin de obtener mejores resultados? ¿Que datos debemos revisar? ¿Que posibilidades adicionales debemos investigar? Por ejemplo, no olvidemos lo que nos permite hacer Big Data: acciones en tiempo real basadas en la transmisión de flujos de información a alta velocidad. Tenemos que definir que parte del negocio necesita acciones en tiempo real para poder influir en las operaciones o en la interacción con el cliente. Una vez que hayamos definido estas acciones en tiempo real, tenemos que asegurarnos de que existan sistemas automatizados o procesos para la realización de dichas acciones y proporcionar mecanismos de recuperación ante fallos en caso de problemas.