\capitulo{8}{Resultados}

En está penúltima sección se exponen las conclusiones derivadas del trabajo. 



\section{Preámbulo}
Los resultados son el remate final a a este estudio, sin embargo, quiero hacer aquí una pequeña aclaración. Los efectos de este estudio realizado por otro conjunto de personas y partiendo de los mismos datos pueden ser diferentes dado que el enfoque que se le da es distinto. Si bien es cierto que se ha encontrado patrones similares con otros trabajos (ver más en la sección 6) no era este el objetivo final. 

En el siguiente diagrama se puede ver el conjunto de desarrollo completo que se ha seguido para llegar a los resultados.

  \imagen{diagramtech}{\footnotesize{Diagrama desarrollo. Fuente: Elaboración propia.}}

\section{Resultados generales obtenidos}
Vamos a realizar aquí una exposición de los resultados que se han obtenido en el estudio.  

\section{Preguntas analíticas}

Aquí se intentará, dado todo el estudio y con ayuda de datos, responder a diversas preguntas de carácter analítico.


-          Los protocolos de anticontaminación se basan en los niveles de los óxidos de nitrógeno. Estos niveles suben por la noche por lo que siempre se espera a última hora para activar dichos protocolos. Es una afirmación correcta ¿

-          Existen contaminantes que puedan estar asociados a temporalidad ?

-          Consideras que las últimas medidas aplicadas por el Ayuntamiento de Madrid durante los últimos años han mejorado la calidad del aire? Que contaminantes? Alguna zona/estación en concreto ? Puedes confirmar esta información contrastando la evolución temporal del número de alertas, avisos y preavisos en las diferentes zonas ?

-          Qué área es la más contaminada ?