\capitulo{8}{Resultados}

En está penúltima sección se exponen las conclusiones derivadas del trabajo. 



\section{Preámbulo}
Los resultados son el remate final a este estudio, sin embargo, quiero hacer aquí una pequeña aclaración. Los efectos de este estudio realizado por otro conjunto de personas y partiendo de los mismos datos pueden ser diferentes dado que el enfoque que se le da es distinto. Por ejemplo yo me he enfocado más en un contaminante, \ce{NO2}, y en éste comparado a los niveles de la OMS \cite{oms_2}. Si bien es cierto que se ha encontrado patrones similares con otros trabajos (ver más en la sección 6) no era este el objetivo final. 

En la figura \ref{diagramtech} se puede ver el conjunto de desarrollo completo que se ha seguido para llegar a las visualizaciones.

  \imagen{diagramtech}{\footnotesize{Diagrama desarrollo. Fuente: Elaboración propia.}\label{diagramtech}}
  
  Los resultados que se exponen en el siguiente apartado son los que se han obtenido de observar, analizar y estudiar esas visualizaciones para dar respuesta a las preguntas analíticas. Una vez con los datos limpios los gráficos o resultados a obtener son tantos como objetivos tengamos en mente.
  

\section{Resultados generales}

Después de un análisis exhaustivo de los datos para los últimos cinco años, podemos concluir con que Madrid tiene un problema de contaminación severo. No solo con los datos en la mano se ratifica esta postura sino que desde la unión Europea 

Además según la OMS la práctica totalidad de la población de España viene respirando aire contaminado, con foco principal en Madrid como punto elevado de concentración de gases tóxicos. Con los datos delante se aprecia una clara mejoría en los datos y en el año 2018 se mejoran los datos respecto del año anterior, 2017.

Una de las primeras dudas que nos pueden surgir a la hora de hacer este tipo de estudios es por qué elegir un tipo de gas u otros. Ya no solo por que la OMS sabe de lo que habla y recoge en su legislación los gases más contaminantes para el ser humano sino basta ver la gráfica promedio de contaminantes (ver figura \ref{promTODOS}) para darse cuenta de los gases que mayores niveles tienen en los medidores son los más perjudiciales.

\imagen{promTODOS}{\footnotesize{Detalle gases tóxicos (2014-2018). Fuente: Elaboración propia.}\label{promTODOS}}

De acuerdo a ese tipo de gráficas promedio por contaminante el incremento de determinados tipo de contaminantes como las \ce{PM2,5} (Partículas Menores de 2,5 \textmugreek m), \ce{PM10} (Partículas Menores de 10 \textmugreek m), \ce{NO} (monóxido de nitrógeno) y \ce{NO2} (dióxido de nitrógeno) se viene relacionando desde hace tiempo con el aumento de ingresos hospitalarios por afecciones respiratorias y cardiovasculares por lo que tiene sentido el estudio de estas particulas.

Más concretamente nos hemos centrado en \ce{NO2}. Algunas apreciaciones interesantes que se han obtenido son por ejemplo que influye la \textbf{estacionalidad}. Los valores se disparan en los períodos comprendidos entre mediados de octubre y mediados de enero. Ver figura \ref{estacionalidad} con la media anual del año 2018 para ver claramente los patrones. Los meses con los menores niveles son junio, julio y agosto. 

\imagen{estacionalidad}{\footnotesize{Media \ce{NO2} en 2018. Fuente: Elaboración propia.}\label{estacionalidad}}

De la misma manera se observan patrones influyentes en el \textbf{día de la semana} y la \textbf{hora del día} . Se diferencian los días laborales de los festivos.  En días laborables existen dos zonas horarias de mayor contaminación correspondiendo éstas a las horas pico de tráfico por la mañana y por la tarde en torno a las 20-21 horas.  Se ve claramente en los días laborables como se producen dos picos claros que representar la entrada y salida del trabajo (ver figura \ref{promedioHora}).  En el caso del fin de semana, la zona más relevante se corresponde con las horas nocturnas, observándose un incremento progresivo desde las 19 horas hasta 24 horas. 

	
\imagen{promedioHora}{\footnotesize{Detalle promedio \ce{NO2} por hora. Fuente: Elaboración propia.}\label{promedioHora}}


La zona del \textbf{interior de la M30} es sin lugar a dudas donde se dan los valores más elevados de \ce{NO2}. También es el lugar dónde más afectados se ven los vehículos a partir del escenario de contaminación número 2. El mayor número de estaciones (más de un 40\%) se concentra en esta zona. De la misma manera el mayor número de valores que superan los 180 \textmugreek g/m3 también se sitúan en esta zona.

En la figura \ref{estaciones} es posible ver por orden de mayor a menor las estaciones dónde se producen los mayores niveles. Las líneas representan la cantidad total de preavisos (mayor 180) en color negro y en color rojo los avisos (mayor 200).

\imagen{estaciones}{\footnotesize{Detalle mayor concentración de \ce{NO2} por estación. Fuente: Elaboración propia.}\label{estaciones}}


Durante los años estudiados se producen grandes amplitudes en los datos aunque se focalizan por zonas como hemos dicho, siendo \textbf{dentro de la M30} donde se dan las mayores concentraciones. Cabe destacar que dentro de esta zona los valores menores se corresponden a la estación de Retiro. Esto podría confirmar que la masa forestal del Parque del Retiro disminuye sensiblemente los valores de contaminación.

Dentro de ésta zona podemos considerar que para todas las estaciones, salvo para la de Retiro, los valores atípicos son aquellos que están ligeramente por encima de 100-110  \textmugreek  g/m3.

Algo que hemos notado también es que el ayuntamiento de Madrid no aplica los protocolos siempre. Desde la página web del ayuntamiento se especifican cuando se aplican los protocolos \cite{mambiente} sin embargo en los datos podemos notar avisos durante la campaña de navidad de 2015 (ver figura \ref{navidad2015}, donde se aprecia claramente que se superan los niveles establecidos por OMS) en los cuales no se han activado los protocolos a pesar de tener avisos. Es posible que esas fechas en las que se produce un mayor número de desplazamientos el ayuntamiento tenga, quizás, algo de mano ancha. 

\imagen{navidad2015}{\footnotesize{Detalle período navidad 2015. Fuente: Elaboración propia.}\label{navidad2015}}

Se podría pensar que activando los protocolos se disminuiría la concentración de \ce{NO2}  (disminución del tráfico) para los días inmediatamente siguientes a la activación. Sin embargo, esto no se aprecia en picos altos durante el invierno. Por lo que se puede tener como hipótesis que parece que no basta el tráfico para la disminución de este gas. Puede deberse a qué durante el invierno un \% de emisiones de este gas provenga del uso de calefactores. Hay que tener en cuenta que, en el centro de las ciudades, que son las que se ven afectadas por los cortes de tráfico, la mayor parte de los edificios cuentan con sistemas de calefacción basadas en la combustión de carbón o gasóleo.

\section{Preguntas analíticas}

Algunas de las preguntas analíticas, marcadas al inicio del proyecto, han sido las siguientes:

\begin{itemize}
	\item \textit{¿La contaminación es un problema real en la ciudad de Madrid?} \textit{¿Supone un grave riesgo para la población?} \textit{¿Son las partículas de Madrid contaminantes de alto riesgo para la salud?}
	
	La principal respuesta a esta pregunta depende de los valores que se toman como referencia. Los valores que dictamina la OMS \cite{oms_2} son mucho más estrictos que los límites legales que establecen las instituciones y por lo tanto más acordes con la protección de la salud. De acuerdo a esos valores claramente Madrid tiene un serio riesgo de contaminación. 
	
	De acuerdo a los límites legales también dado que se ven superados en numerosas ocasiones por lo que los problemas para la salud existen. Además existen estudios \cite{ambiente_contaminacion_estudio} que ratifican el empeoramiento de la salud como causa directa de respirar este tipo de aires contaminados.
	
	Tal y como hemos estudiado en la ciudad de Madrid  los contaminantes que más incidencia presentan por regularidad y niveles son el dióxido de nitrógeno, el ozono  y las partículas \ce{PM10} y \ce{ PM2,5}. Los peores registros se lo llevan las estaciones que se encuentran dentro de la M30, por lo tanto las personas mas expuestas son las que se encuentran en esta zona pero por supuesto no las únicas.
	
	Madrid es de las pocas ciudades europeas en las que se sigue registrando el incumplimiento del valor límite horario de \ce{NO2}. Además desde Europa nos sigue llegando multas \cite{bruselas_} por incumplimiento del reglamento por lo que realmente parece que es un problema real en el que no se están tomando las medidas adecuadas. Si bien es cierto que no presenta niveles extremos como pueden darse en ciudades asiáticas, de cara al futuro convendría extremar las precauciones.
	
	Para terminar con la pregunta, y algo en relación al tema aunque fuera de la pregunta, según un último estudio \cite{calderon_garciduenas_combustion_2019} el corazón de las personas jóvenes que viven en grandes ciudades contienen miles de millones de partículas contaminantes. Estudio de la Universidad de Lancaster (Reino Unido), y que podría suponer la primera demostración de la ya conocida relación entre la calidad del aire y las enfermedades relacionadas con este órgano. El estudio se centra en diferentes partículas, como por ejemplo partículas derivadas de la exposición prolongada a \ce{NO2} u \ce{O3}, pero está desarrollado en la ciudad de México. Sería interesante ver un estudio de este tipo en otras ciudades Europeas, como Madrid.
	
	\item \textit{Los protocolos de anticontaminación se basan en los niveles de los óxidos de nitrógeno. Estos niveles suben por la noche por lo que siempre se espera a última hora para activar dichos protocolos. ¿Es una afirmación correcta?}
	
	Es una afirmación correcta con los datos en la mano. Pero vamos a analizarla un poco más. Las limitaciones, de diversas índoles dependiendo del escenario, comienzan a aplicarse partir de las 6:00 horas del día de entrada en vigor y termina cuando cesa el episodio de contaminación. Cuando este episodio cesa significa que una determinada estación no establece un preaviso, aviso o alerta pero eso no significa que la contaminación haya desaparecido. Normalmente 
	

	
	El problema que hemos detectado es que los protocolos de contaminación tan solo se aplican a determinadas zonas de Madrid. O al menos así se recogen en los informes públicos. Es decir, existen estaciones cuyas lecturas no activan, directamente, protocolos para restringir el tráfico u otras medidas. Los motivos son claros: al encontrarse fuera de la M-30, muchas estaciones que sobrepasan los límites establecidos quedan excluidos de los protocolos. Las alarmas pueden saltar por episodios de contaminación durante horas
	
	 Si nos paremos a pensar en factores externos como por ejemplo el clima, se dan casos en los que el centro de Madrid registra niveles seguros sin embargo todas las estaciones que lo rodean presentan preavisoscon a tasas insalubres sin que salten las alarmas.
	
	\item \textit{¿Existen contaminantes que puedan estar asociados a temporalidad?}
	
	Sin duda la respuesta es sí, un ejemplo muy claro es el del \ce{NO2}. Los efectos de este gas son producidos en su mayoría por la combustión. El \% mayor proviene de los coches pero durante el invierno existen multitud de hogares en la capital que todavía funcionan con viejas estufas de combustión por lo que los niveles de contaminación de las mediciones se ven afectado también por este tipo de episodios.
	
	\item \textit{¿Consideras que las últimas medidas aplicadas por el Ayuntamiento de Madrid durante los últimos años han mejorado la calidad del aire? ¿Qué contaminantes? ¿Alguna zona/estación en concreto? ¿Tiene sentido Madrid Central?}
	
	Los contaminantes más problemáticos en España durante los años se repiten de manera constante y son las	partículas en suspensión (\ce{PM10} y \ce{PM2,5}), el dióxido de nitrógeno (\ce{NO2}), el ozono troposférico (\ce{O3}) y el dióxido de azufre (\ce{SO2}). 
	
	De los últimos 6 años los peores registros se los llevan sin lugar a dudas 2017 pero en general se tienen en cuenta las medidas. Sin lugar a dudas la más polémica del pasado año ha sido Madrid Central. Si no estamos equivocados, el ayuntamiento de Madrid fue la primera ciudad en activar un protocolo \cite{protocolo} de estas características en nuestro país. 

	El 1 de julio de 2019 la corporación municipal de la ciudad cesó el conocido plan de la legislatura anterior para prevenir la contaminación en el centro de la ciudad. El nuevo ayuntamiento permite ahora entrar a cualquier automóvil para circular a sus anchas por Madrid Central, sin ningún tipo de
	limitación ni consecuencias. 
		
		Los datos de contaminación por \ce{NO2} registrados durante el mes de junio de 2019 en Madrid, muestran dos fases bien diferenciadas: una fase inicial en la que los valores de \ce{NO2} alcanzados fueron muy similares a los correspondientes a los meses de abril y mayo, es
		decir, con registros históricamente bajos, tanto en la estación ubicada dentro del perímetro de Madrid Central (recordemos es \textit{Plaza del Carmen}), como en el valor medio de toda la red; y una segunda fase, coincidiendo con el anuncio de la
		suspensión de las multas en Madrid Central a partir del 1 de julio, en la que se observa un rápido empeoramiento de los valores de \ce{NO2} registrados en la estación Plaza del Carmen, que se hace muy intensa en la última semana
		del mes. 
		
		En la figura \ref{junjul2019} tenemos exactamente las gráficas del mes de \textbf{Junio}, en la parte izquierda, con Madrid central aún en vigor; y del mes de \textbf{Julio}, en la parte derecha y con el fin de Madrid central. Se aprecian claramente los picos de los que hemos hablado en el párrafo anterior. En azul se aprencian los máximos y en naranja la media diaria de \ce{NO2} en cada mes.
		
		\imagen{junjul2019}{\footnotesize{Detalle \ce{NO2} Junio/Julio 2019. Fuente: Elaboración propia.}\label{junjul2019}}
		
		En la figura \ref{plazacarmen} se puede ver la evolución en años anteriores. Recordemos que Madrid Central entra en vigor a finales de 2018. Si comparamos la media de toda la red con respecto de la estación que se encuentra dentro de esta delimitación los valores de ésta última siempre han sido superiores hasta la entrada en vigor de las nuevas medidas. Éstas nuevas medidas se empiezan a notar a partir de enero de 2019.
		
		\imagen{plazacarmen}{\footnotesize{Detalle estación \textit{Plaza del carmen}. Fuente: Elaboración propia.}\label{plazacarmen}}
		
		Si por ejemplo comparamos entre \textit{Junio 2018} y \textit{Junio 2019}. En 2019 se registró 26 \textmugreek g/m3, 12 menos (reducción del 32 \%) que el registrado en 2018 (38 \textmugreek g/m3).
		
		 Realmente todavía es pronto para saber las consecuencias reales del cese de la prohibición dado que con un solo mes de comparativa, tan solo julio 2019 está disponible por el momento, resulta apresurado realizar estimaciones con acierto. También hay que tener en cuenta que nos encontramos en los meses de menores tasas por lo que será necesario esperar a los meses con valores más altos para determinar el impacto.
		 
		  Todavía ni los propios madrileños tienen claro si se aplican o no sanciones dentro de la zona cero ya que el juzgado de lo Contencioso-Administrativo número 7 de Madrid emitió una nueva resolución \cite{auto_judicial} que ratificaba la reactivación de multas en Madrid Central al dar el visto bueno a la medida cautelar solicitada por Greenpeace. Lo que sí resulta verídico es que desde Europa se plantean las sanciones administrativas \cite{20minutos_europa_2019} pertinentes en caso de no tomar medidas ante la calidad del aire de la ciudad de Madrid. 
		 
		 Lo que sí parecen los datos constatar es la mejora de la estación respecto de años anteriores durante el tiempo que Madrid Central seguía en vigor.
		
			

	
	\item \textit{¿Qué área es la más contaminada?}
	
	La área más contaminada es la del interior de la M30, y dentro de ella \textit{Escuelas Aguirre} tiene tendencia a marcar los mayores valores. Las estaciones de Pza. Fernández Ladreda (Suroeste) y las estaciones de Escuelas Aguirre (Interior M30) son las estaciones que suelen marcan los máximos. En relación a esta última podemos ver en la figura \ref{EscuelasAguirre} como claramente se superan los límites marcados por la OMS \cite{oms_2} en cuanto a los contaminantes partículas en suspensión (\ce{PM10} y \ce{PM2,5}) y el dióxido de nitrógeno (\ce{NO2}).
	
	\imagen{EscuelasAguirre}{\footnotesize{Detalle estación \textit{Escuelas aguirre}. Fuente: Elaboración propia.}\label{EscuelasAguirre}}
	
	Las diferencias entre el interior y el exterior de la M-30 son notables. En enero se captan los niveles más altos de todo el año y en los meses de verano, en los que la capital reduce su población debido a las vacaciones, estos niveles son reducidos. 
	
	
	\item \textit{¿Qué está ocurriendo en el año en curso 2019?} \textit{¿Demuestran los datos que son útiles las medidas implantadas?}
	
	El año en curso sigue la tendencia de mejora de los anteriores aunque se espera ver como va a afectar la medida del cese de Madrid Central. En la figura \ref{NO220192} se muestra la media de \ce{NO2} (en color naranja) durante los meses de enero a julio. La línea en color azul representa los máximos registrados.
	
	De lo que llevamos de año los meses de enero y febrero vienen tienendo valores similares a los de años anteriores en los mismos meses. En enero solo 3 estaciones redujeron la media respecto a otros años. Durante febrero la estación de Plaza del Carmen, la única ubicada dentro del área de Madrid Central, fue por primera vez en la que más disminuyó la contaminación respecto de otros años.
	
	\imagen{NO22019}{\footnotesize{Detalle \ce{NO2} 2019. Fuente: Elaboración propia.}\label{NO220192}}
	
	Tal y como es posible ver en la gráfica \ref{NO220192} en Marzo comienza un punto de inflexión. El tiempo comienza a variar ligeramente en esta época del año y es también el período (16 de marzo) donde se comienzan a imponer sanciones a los conductores que no respeten los protocolos. 
	
	Durante el mes de abril, en todas las estaciones salvo en dos (Juan Carlos I y Escuelas Aguirre), los niveles de contaminación disminuyeron respecto a años anteriores.
	
	En el mes de mayo todas las estaciones de medición acumularon resultados positivos; es decir, por primera vez desde la entrada en vigor de Madrid Central la ciudad tenía una calidad de aire por encima de la media del periodo de los cinco años anteriores.
\end{itemize}



	