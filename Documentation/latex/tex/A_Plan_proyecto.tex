\apendice{Plan de Proyecto Software}

\section{Introducción}\label{introduccion-plan}
En este capitulo se detalla la planificación del proyecto. 

La planificación temporal del proyecto, o lo que es lo mismo, la elaboración del calendario o programa de tiempos, consiste en una representación gráfica de todas las actividades del proyecto necesarias para producir el resultado final que se desea.

Se han utilizado metodologías ágiles para el desarrollo del proyecto y de este modo, se ha realizado un desarrollo dividido en iteraciones. Terminada una iteración empezaba la siguiente y se agregaban a las tareas planeadas las que no habían sido completado de la iteracción precedente. Las iteraciones del proyecto, los sprints, estaban pensadas para durar unos \textbf{quince días} aproximadamente. No obstante, hay alguna excepción en la que la iteración duró más tiempo, o menos. También existe alguna demora entre algún sprint debido a que tenía demasiada carga de las asignaturas o demasiada carga laboral.

\section{Planificación temporal}\label{planificacion-temporal}
Desde el comienzo del proyecto se planteó utilizar una metodología ágil como
\emph{Scrum} para la gestión del proyecto. Aunque no se ha seguido al 100\% la
metodología al tratarse de un proyecto académico, sí que se ha aplicado
en líneas generales una filosofía ágil y metódica. La diferencia fundamental radica en que esta metodología esta pensada para equipos y no para individuos.

\subsection{Desarrollo con Scrum}

Scrum es un marco de trabajo que define un conjunto de prácticas y roles, y que puede tomarse como punto de partida para definir el proceso de desarrollo que se ejecutará durante el proyecto. Indicar aquí que se ha utlizado terminología en inglés dado que por convenio se siguen estos términos en cualquier organización. De la misma manera las tareas también se han descrito en inglés. 

Recordemos que Scrum no define como tal un método o herramienta para llevar el seguimiento del trabajo realizado. Scrum solo propone una serie de buenas prácticas por lo que el seguimiento se puede realizar con una simple hoja de excell, cuaderno o con herramientas de gestión especializadas. Yo he elegido el \textit{project board} de github dentro de mi repositorio del trabajo fin de máster.

Los roles o actores principales en Scrum serían los siguientes: 
\begin{itemize}	
	\item \textit{Scrum Master}: su trabajo prioritario es eliminar los obstáculos que impiden que el equipo no alcance el objetivo del \textit{sprint}. Esta persona no es el líder del equipo sino el nexo entre todos ellos. El autor del trabajo asume este rol.
	\item \textit{Product Owner}: representa a los \textit{stakeholders}\footnote{En toda organización, además de sus propietarios, participan diversos actores claves y grupos sociales que están constituidos por las personas o entes que, de una manera y otra, tienen interés en el desempeño de una empresa porque están relacionadas, bien directa, bien indirectamente, con ella. Fuente Wikipedia. \url{https://es.wikipedia.org/wiki/Parte_interesada_(empresas)}}, se asegura de que el equipo trabaje deforma adecuada desde la perspectiva de éstos. En un entorno real sería el encargado de escribir historias de usuario, priorizarlas y colocarla en el \textit{Product Backlog}. En este caso que se describe, recordemos en un entorno académico, el tutor académico sería el más indicado para asumir este rol. 
	\item \textit{Team} (equipo): en un entorno real sería el equipo que conforman todos los profesionales de diferentes ramas de conocimiento. El objetivo principal es el de entregar el producto. En este caso el autor asume el rol de equipo.
\end{itemize}

A continuación se describre el ciclo de desarrollo. Al inicio del proyecto el \textit{Product Owne}r debe definir los requisitos que serán los objetivos a cumplir. Éstos quedan reflejados y ordenados por orden de importancia en el \textit{Product Backlog}. Durante los sprints cada uno de estos objetivos serán subdivididos en tareas más pequeñas. En la figura \ref{backlog} se puede ver un ejemplo del \textit{Backlog}. 

\imagen{backlog}{\footnotesize{Detalle \textit{Product Backlog}.}\label{backlog}}

Al inicio de cada Sprint se extraen una serie de tareas de los elementos del \textit{Product Backlog} que conforman el \textit{Sprint Backlog}, las tareas son elegidas por el equipo. Estas tareas deben ser completadas durante el ciclo y no es posible añadir tareas nuevas a este \textit{Sprint} durante el tiempo que se está realizando el mismo.

\begin{itemize}	
	\item \textit{Product Backlog}: en un entorno real, consistiría en un listado de historias de usuario, obtenidas con el cliente, que se irán incorporando al producto a partir de incrementos sucesivos. En el proyecto descrito aquí éste se ve reflejado en las tareas (\textit{issues} en github) que se van creado a medida que son necesarias. Estás tareas se asignan después a cada sprint.
	\item \textit{Sprint}: una de las bases de los proyectos ágiles es el desarrollo mediante las iteraciones incrementales. En Scrum a cada iteración se le denomina \textit{Sprint}. En el caso concreto de este trabajo, excepto el último, todos tienen una duración de quince días.
\end{itemize}


A continuación se describen las reuniones que se realizan en el marco de desarrollo de trabajo \textit{Scrum}. 
\begin{itemize}	
	\item \textit{Daily Meeting}: se realiza una reunión breve diaria en la que cada miembro del equipo cuenta el estado en el que se encuenta la parte del proyecto que está realizando. En el caso del proyecto que se describe en esta documentación deberá ser una reflexión personal.
	\item \textit{Planning Meeting}: se realiza al inicio de cada \textit{Sprint} con el objetivo de escoger las tareas necesarias a realizar. Recodemos que se deben escoger del \textit{Product Backlog}.
	\item \textit{Review Meeting}:  esta reunión se realiza con el cliente al final de cada \textit{Sprint}, suele tener una duración máxima de cuatro horas y se revisa las historias de usuarios que han finalizado y las que no. Además se suele realizar una pequeña demo.
	\item \textit{Retrospective Meeting}: esta reunión también se realizala finalizar el \textit{Sprint}. Todos los miembros del equipo expresan sus opiniones sobre la iteracción finaliza. El objetivo de esta reunión es realizar una mejora continua del proceso. 
\end{itemize}


A continuación se describen los diferentes \emph{sprints} que se han realizado de manera detallada así como la planificación final en forma de diagrama de gant.

%PlanificacionInicial-----------------------------
%\subsubsection{Planificación inicial}

 %\hl{Incluir aqui diagrama ganntttt INICIAL}

%\imagen{planificacioninicial}{\footnotesize{Planificación temporal inicial.}\label{planificacioninicial}

%Previo-----------------------------
\subsubsection{Mayo/Junio 2019}\label{previo}
El trabajo fin de máster fue elegido a finales de marzo de 2019. Debido a la intesa carga de trabajo del máster y a que la actividad del autor no se limita únicamente al ámbito académico sino que también desempeña un trabajo a tiempo completo durante la semana, no se pudo empezar a desarrollar una actividad completa hasta el mes de junio. 
En la figura \ref{scrumBoard} se puede ver el \textit{board} al inicio.

\imagen{scrumBoard}{\footnotesize{Detalle \textit{Scrum board}.}\label{scrumBoard}}

%Sprint0-----------------------------
\subsubsection{Sprint 0: 15/06/2019 - 01/07/2019}\label{sprint0}
\begin{itemize}
	\item[$\ast$] \textbf{Duración}:  15 días
	\item[$\ast$] \textbf{Descripción \textit{Sprint}}: Este primer \textit{Sprint} no se puede considerar como tal, de ahí el número cero, ya que no se estaba desarrollado al cien por cien el trabajo. La prioridad eran las asignaturas. Durante estos primeros quince días se crearon las crear tareas y se llenó el \textit{backlog}. De la misma manera también mucho tiempo dedicado a leer durante este \textit{Sprint}, con la idea de comenzar a entender el funcionamiento de la parte administrativa.
\end{itemize}
%Sprint1-----------------------------
\subsection{Sprint 1: 01/07/2019- 15/07/2019 (ver figura \ref{sprint1_img})}\label{sprint1}
\begin{itemize}
\item[$\ast$] \textbf{Duración}:  15 días
\item[$\ast$] \textbf{Descripción \textit{Sprint}}: En este primer \textit{Sprint} efectivo se comienza a trabajar en hacer un criba sobre todo lo investigado, además se comienza también a crear la documentación, repositorios y a hacer un análisis preliminar de los datos con los que vamos a trabajar.
\end{itemize}
\begin{enumerate}
	\item \textbf{\textit{Planning meeting}}
	\begin{description}
		\item[Reseña:] La primera reunión se fijan las tareas que se van a realizar este \textit{Sprint} y se concuerda con el tutor académico lo que tenemos en mente para comenzar con el desarrollo.
	\end{description}
	\item \textbf{\textit{Sprint planning}}
	\item[$-$] En esta reunión se establecen la tareas (\textit{user stories}) a desarrollar durante el \textit{Sprint}. 
	\begin{itemize}
		\item Crear repositorio.
		\item Investigar tecnologías principales.
		\item Analizar datos.
		\item Discutir propuesta con tutor académico.
		\item Discutir propuesta con tutor empresa.
		\item Terminar investigación sobre legislación.
		\item Terminar investigación científica sobre calidad del aire.
		\item Documentación: Introducción.
		\item Documentación: Objetivos.
	\end{itemize}
	\item \textbf{\textit{Retrospective meeting}}
	\begin{itemize}
		\item ¿Qué ha ido bien?
		\begin{itemize}
			\item Toda la parte de investigación tanto sobre legislación como científica.
			\item Reuniones muy productivas con los tutores para aclarar diversas dudas de índole diferente.
			\item Comienzo con la memoria de manera constante y paralela a las demás partes.
		\end{itemize}
		\item ¿Qué dificultades hemos encontrado?
		\begin{itemize}
			\item Debido a la enorme cantidad de información teórica que he recabado me va a ser complicado volcarla toda rápida
			\item Exiten numerosísimas opciones en cuanto a tecnologías. La oferta en el mercado es muy amplia y ello me crea  incertidumbre al respecto.
		\end{itemize}
	\end{itemize}	
\end{enumerate}
\imagen{sprint1}{\footnotesize{Detalle Sprint 1.}\label{sprint1_img}}

%Sprint2-----------------------------
\subsection{Sprint 2: 15/07/2019- 31/07/2019(ver figura \ref{sprint2_img})}\label{sprint2}
\begin{itemize}
\item[$\ast$] \textbf{Duración}:  15 días
\item[$\ast$] \textbf{Descripción \textit{Sprint}}: En este \textit{Sprint} se ha enfocado más a comprender, analizar e interpretar toda la información de la que se disponía. Ha sido un sprint de escribir mucho, asimilación y análisis antes de volver a la parte técnica. A veces es necesario tener claro todos los conceptos para saber como avanzar. 
\end{itemize}
\begin{enumerate}
\item \textbf{\textit{Planning meeting}}
\begin{description}
	\item[Reseña:] En esta reunión se fijan las tareas que se van a realizar este \textit{Sprint}. Además se añaden algunas nuevas al backlog como las referidas a la parte técnica que se realizarán en scripts sucesivos y que vienen derivadas de las necesidades de éste \textit{Sprint}.
\end{description}
\item \textbf{\textit{Sprint planning}}
	\item[$-$] En esta reunión se establecen la tareas (\textit{user stories}) a desarrollar durante el \textit{Sprint}. 
\begin{itemize}
	\item Creación de scripts y arquitectura del TFM.
	\item Preprocesamiento y limpieza de datos.
	\item Documentación: conceptos teóricos (Big Data).
	\item Documentación: conceptos teóricos (Legislación).
	\item Documentación: conceptos teóricos (Método análisis).
	\item Documentación: conceptos teóricos (Análisis científico).

\end{itemize}
\item \textbf{\textit{Retrospective meeting}}
\begin{itemize}
	\item ¿Qué ha ido bien?
	\begin{itemize}
		\item Con este sprint se ha comprendido una de las partes más importantes del trabajo: la parte científica.
		\item Toda la parte de preprocesamiento de datos.
	\end{itemize}
	\item ¿Qué dificultades hemos encontrado?
	\begin{itemize}
		\item Debido a la extensión de los conceptos teóricos no he podido progresar todo lo que me hubiera gustado en la parte técnica. Será necesario refactorizar código en \textit{sprints} sucesivos.
	\end{itemize}
\end{itemize}	
\end{enumerate}
\imagen{sprint2}{\footnotesize{Detalle Sprint 2.}\label{sprint2_img}}

%Sprint3-----------------------------
\subsection{Sprint 3: 31/07/2019 - 15/08/2019 (ver figura \ref{sprint3_img})}\label{sprint3}
\begin{itemize}
	\item[$\ast$] \textbf{Duración}:  15 días
	\item[$\ast$] \textbf{Descripción \textit{Sprint}}: Este tercer \textit{Sprint} se ha dirigido más a la parte técnica. Tanto seguir y terminar tareas del \textit{sprint }anterior como en la parte de visualización.
\end{itemize}
\begin{enumerate}
	\item \textbf{\textit{Planning meeting}}
	\begin{description}
		\item[Reseña:] En esta reunión se han escogido las tareas a realizar en este sprint.
	\end{description}
	\item \textbf{\textit{Sprint planning}}
		\item[$-$] En esta reunión se establecen la tareas (\textit{user stories}) a desarrollar durante el \textit{Sprint}. 
	\begin{itemize}
		\item Visualización de datos en \textit{Python}.
		\item Visualización de datos en \textit{R}.
		\item Refactorización de código.
		\item Documentación: aspectos relevantes del desarrollo.
		\item Documentación: técnicas y herramientas.
		\item Documentación: trabajos relacionados.
		\item Enviar copia a tutores de los puntos de la memoria desarrollados en el \textit{sprint} anterior.
	\end{itemize}
	\item \textbf{\textit{Retrospective meeting}}
	\begin{itemize}
		\item ¿Qué ha ido bien?
		\begin{itemize}
			\item Este \textit{sprint} se ha enfocado en trabajar las visualizaciones tanto en \textit{Python} como en \textit{R}. 
			\item Se ha avanzado mucho en la memoria y además se ha enviado una primera copia a los tutores para su revisión.
		\end{itemize}
		\item ¿Qué dificultades hemos encontrado?
		\begin{itemize}
			\item Ninguna dificultad destacada. 
		\end{itemize}
	\end{itemize}	
\end{enumerate}
\imagen{sprint3}{\footnotesize{Detalle Sprint 3.}\label{sprint3_img}}

\subsection{Sprint 4: 15/08/2019 - 31/08/2019 (ver figura \ref{sprint4_img})}\label{sprint4}
\begin{itemize}
	\item[$\ast$] \textbf{Duración}:  15 días
	\item[$\ast$] \textbf{Descripción \textit{Sprint}}: Este cuarto \textit{Sprint} se ha dirigido más a la parte de la visualización final aunque sin olvidar la documentación. Se ha modificado también la documentación en base al primer envío a los tutores con las mejoras propuestas por éstos.
\end{itemize}
\begin{enumerate}
	\item \textbf{\textit{Planning meeting}}
	\begin{description}
		\item[Reseña:] En esta reunión se han escogido las tareas a realizar en este \textit{sprint}.
	\end{description}
	\item \textbf{\textit{Sprint planning}}
	\item[$-$] En esta reunión se establecen la tareas (\textit{user stories}) a desarrollar durante el \textit{Sprint}. 
	\begin{itemize}
		\item Visualización de datos con PowerBI.
		\item Investigación de uso de la herramienta software \textit{tinybird}.
		\item Discusión de resultados.
		\item Documentación: plan de proyecto software (1/2).
		\item Documentación: entorno experimental.
		\item Documentación: resultados.
	\end{itemize}
	\item \textbf{\textit{Retrospective meeting}}
	\begin{itemize}
		\item ¿Qué ha ido bien?
		\begin{itemize}
			\item Este \textit{sprint} se ha dirigido a trabajar en la parte de la visualización final.
			\item Se ha avanzado en la memoria respecto de las partes finales de la misma. 
			\item Al introducirse una nueva herramienta (tinybird) se ha estado investigando su funcionamiento.
		\end{itemize}
		\item ¿Qué dificultades hemos encontrado?
		\begin{itemize}
			\item La parte de la visualización final se ha comido demasiado tiempo por lo que todas las tareas de documentación no han podido finalizarse tal y como se esperaba.
			\item Por retrasos en la planificación no ha sido posible enviar la segunda parte de la memoria a los tutores
		\end{itemize}
	\end{itemize}	
\end{enumerate}
\imagen{sprint4}{\footnotesize{Detalle Sprint 4.}\label{sprint4_img}}

\subsection{Sprint 5: 31/08/2019- 06/09/2019 (ver figura \ref{sprint5_img})}\label{sprint5}
\begin{itemize}
	\item[$\ast$] \textbf{Duración}:  6 días
	\item[$\ast$] \textbf{Descripción \textit{Sprint}}: Este \textit{Sprint} de corta duración se ha dedicado a revisión.
\end{itemize}
\begin{enumerate}
	\item \textbf{\textit{Planning meeting}}
	\begin{description}
		\item[Reseña:] En esta reunión se han escogido las tareas a realizar en este \textit{sprint}. Se han añadido las incompletas del sprint anterior.
	\end{description}
	\item \textbf{\textit{Sprint planning}}
	\item[$-$] En esta reunión se establecen la tareas (\textit{user stories}) a desarrollar durante el \textit{Sprint}. 
	\begin{itemize}
		\item Visualización de datos con PowerBI.
		\item Investigación de uso de la herramienta software \textit{tinybird}.
		\item Página web resultados.
		\item Documentación: plan de proyecto software (2/2).
		\item Documentación: conclusiones.
		\item Enviar segunda copia de la memoria a los tutores
	\end{itemize}
	\item \textbf{\textit{Retrospective meeting}}
	\begin{itemize}
		\item ¿Qué ha ido bien?
		\begin{itemize}
		\item Se han finalizado los objetivos propuestos.
		\end{itemize}
		\item ¿Qué dificultades hemos encontrado?
		\begin{itemize}
		\item Quizás hemos estado con tiempo justo para terminar todas las tareas con solvencia.
		\end{itemize}
	\end{itemize}	
\end{enumerate}
\imagen{sprint5}{\footnotesize{Detalle Sprint 5.}\label{sprint5_img}}

%PlanificacionFinal-----------------------------
\subsubsection{Planificación final}

En la siguiente figura (\ref{planificacionfinal}) es posible ver un diagrama de gantt con las tareas.

\imagen{planningGantt}{\footnotesize{Planificación temporal final.}\label{planificacionfinal}}


\section{Viabilidad legal}\label{estudio-viabilidad}

La viabilidad legal se centra principalmente en el estudio de las licencias software utilizadas. Realizaremos una tabla resumen sobre las licencias. Indicar aquí que al tratarse de un estudio analítico se ha intentado seguir e utilizar software de carácter \textit{open source}.

\begin{table}[H]
	\begin{center}
		\begin{tabular}{|l|l|}
			\hline
			Dependencias & Licencia \\
			\hline \hline
			\emph{Github pages} & \href{https://opensource.org/licenses/MIT}{MIT}\\ \hline
			\emph{R} & \href{https://en.wikipedia.org/wiki/GNU_General_Public_License#Version_2}{GNU}\\ \hline
			\emph{Python} & \href{https://es.wikipedia.org/wiki/Python_Software_Foundation_License}{PSFL}\\ \hline
			\emph{Vega} & \href{https://github.com/vega/vega/blob/master/LICENSET}{BSD-3 Clause}\\ \hline
			
		\end{tabular}
		\caption{Tabla resumen-licencias.}
		\label{tabla:licencias}
	\end{center}
\end{table}



Una licencia software es un contrato entre el autor o titular de los derechos de explotación o distribución y el usuario consumidor, usuario profesional o empresa, para la utilización del software cumpliendo una serie de términos y condiciones establecidas dentro de sus cláusulas. Todo el software que empleo es su amplia mayoría es libre ya que la mayoría de código viene o bien de desarrolladores \emph{amateur} o es libre desde su creación. Emplean por tanto las licencias descritas en la tabla anterior por lo que diferencia entre ellas varían en términos como el nombramiento del autor o la garantía , son licencia comunes en software libre \cite{githublicense} , además github provee una página para ayudarte en la elección de tu licencia \citep{githubchoose}. La menos permisiva de entre las descritas es \emph{MIT} \citep{mit}  y la más es la licencia \emph{Apache-2.0} \cite{apache}.

\imagen{licencias}{\footnotesize{Gráfico licencias Open Source in GitHub. Fuente:  \url{https://cartograf.net}.}}

Para el proyecto de manera general (está alojado en github de forma libre) se ha elegido la licencia  \href{https://github.com/aguadotzn/TFM_MUINBDES/blob/master/LICENSE}{MIT}.

Para la documentación se ha elegido una licencia Creative commons, en concreto se ha elegido la \emph{Attribution-NonCommercial 4.0 International (CC BY-NC 4.0)}.

\imagen{creativecommons}{\footnotesize{Licencia Creative Commons.}}\label{commons}


\section{Links Importantes}\label{links_final}

-Pagina web
-visualizacion
-mapa de carto
-mapa de carpetas

 \hl{Rellenar antes de entregar}



